\documentclass{article}
\title{OS161 Project 1 Executive Summary}
\author{Noah Weiner}
\begin{document}
\date{}
\maketitle

\paragraph{Collaborators} I worked mostly alone, although I did have guidance from from the Pearls in Life blog and you (Neil). After I had finished I also helped out Eli, Michelle, Tai, and Jonathan a bit.

\paragraph{What I completed} I implemented locks and condition variables, and created different thread counter synchronization tests using using locks, spinlocks, and no synchronization primitives. I demonstrated that each of these worked as expected: the lock/spinlock tests produced correct results while the unsafe counter failed to, and the spinlock test ran much faster than the lock test, although the unsafe test was faster than both.

\paragraph{What I didn't complete} I would have liked to write more tests to ensure that the locks will never fail, and to check that the condition variables work - I don't think I wrote a test using them. Also, I implemented re-entrant locks in theory but didn't write any code to test that out.

\paragraph{What I learned} I learned a lot about how to write kernel code. OS161 uses a set of conventions that mimic classes by associating a set a functions with a struct, something that I picked up as I progressed through the lab. I also became much more familiar with how synchronization primitives work, such as using spinlocks to provide synchronization for other synchronization primitives.

\end{document}
